\documentclass[11pt,english]{article}
\usepackage[latin9]{inputenc}
\usepackage[letterpaper]{geometry}
\geometry{verbose,tmargin=1in,bmargin=1in,lmargin=1in,rmargin=1in}
\usepackage{babel}
\usepackage{amsmath}
\usepackage{amssymb}
\usepackage{capt-of}
\usepackage{graphicx}
\usepackage[usenames,dvipsnames]{color}
\usepackage{latexsym}
\usepackage{xspace}
\usepackage{pdflscape}
\usepackage[hyphens]{url}
\usepackage[colorlinks]{hyperref}
\usepackage{enumerate}
\usepackage{hyperref}
\usepackage{float}
\usepackage{array}
\usepackage{tikz}
\usetikzlibrary{shapes}
\usepackage{algorithm2e}
\setcounter{MaxMatrixCols}{20}

\newcommand{\rthree}{\mathbb{R}^3}
\title{CIS 580 Homework 1 \\
Due: Monday}
 \author{Gabrielle Merritt}
 
\date{}

\begin{document}
\maketitle
\section*{ Camera Model }
\subsection*{CCD of iPhone 6}
The iPhone uses a sony isx014 cmos sensor with a 4.6 mm diagonal. Most likely dimensions are close to 3.67 mm x  2.76 mm.
\linebreak source: \url{ http://www.sony.net/Products/SC-HP/cx_news_archives/img/pdf/vol_70/isx014.pdf}
\subsection*{Sensor Pixel Resolution}
Pixel resolution is approximately 8.08 Megapixels ~ 7.21 pixels per $micron^2$
and pixel size $ =1.12 micron square$ 
\subsection*{Focal Length}
iPhone focal length for the iPhone6 is 29 mm
\subsection*{Measurement} 
TODO: 
\section*{Locate camera optical center by converging lines}



	
\end{document}
